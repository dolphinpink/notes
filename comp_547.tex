% !TeX spellcheck = en_GB

\documentclass[11pt]{article}
\usepackage{amsmath, alltt, amssymb, xspace, times, epsfig}
\usepackage{amssymb}

\setlength{\evensidemargin}{0in} \setlength{\oddsidemargin}{0in}
\setlength{\textwidth}{6.5in} \setlength{\textheight}{8.5in}
\setlength{\topmargin}{0in} \setlength{\headheight}{0in}

\newcommand{\Zn}{\ensuremath{\mathbb{Z}_n}\xspace}
\newcommand{\Zns}{\ensuremath{\mathbb{Z}_n^*}\xspace}
\newcommand{\Zp}{\ensuremath{\mathbb{Z}_p}\xspace}
\newcommand{\Zps}{\ensuremath{\mathbb{Z}_p^*}\xspace}
\newcommand{\Zq}{\ensuremath{\mathbb{Z}_q}\xspace}

\newcommand{\smod}{\ensuremath{\mathrm{mod}\;}}

\newcommand{\DD}{\ensuremath{\mathcal{D}}\xspace}

\newcommand{\Gen}{\ensuremath{\mathsf{Gen}}\xspace}
\newcommand{\Enc}{\ensuremath{\mathsf{Enc}}\xspace}
\newcommand{\Dec}{\ensuremath{\mathsf{Dec}}\xspace}

\renewcommand{\Pr}{\ensuremath{\mathbf{Pr}}\xspace}

\newcommand{\PT}{\ensuremath{\mathsf{PT}}\xspace}
\newcommand{\CT}{\ensuremath{\mathsf{CT}}\xspace}
\newcommand{\Key}{\ensuremath{\mathsf{Key}}\xspace}
\newcommand{\PrivK}{\ensuremath{\mathsf{PrivK}}\xspace}
\newcommand{\eav}{\ensuremath{\mathsf{eav}}\xspace}
\newcommand{\mult}{\ensuremath{\mathsf{mult}}\xspace}

\newcommand{\CC}{\ensuremath{\mathcal{C}}\xspace}
\newcommand{\KK}{\ensuremath{\mathcal{K}}\xspace}
\newcommand{\MM}{\ensuremath{\mathcal{M}}\xspace}
\newcommand{\AAA}{\ensuremath{\mathcal{A}}\xspace}

\newcommand{\sur}[1]{\ensuremath{^{\textrm{#1}}}}
\newcommand{\sous}[1]{\ensuremath{_{\textrm{#1}}}}

\newcommand{\ee}[1]{\ensuremath{#1}}

\newcommand{\note}[1]{\textbf{Note for TA:} #1}\xspace

\begin{document}
%\author{Yiwei Xia}
%\date{}
%\maketitle

\thispagestyle{empty}

\noindent \textbf{COMP 547}
\begin{center}
{\LARGE notes}
\end{center}

\begin{description}

 \item[Block Ciphers] A block cipher is just a keyed function \textit{F} : \ee{\{0,1\}^\textit{l} \times \{0,1\}^\textit{l} \rightarrow \{0,1\}^\textit{l}}  that is a permutation (bijection) of the original block
 
 \item[SPN] \hfill
 \begin{itemize}
 	\item confusion diffusion paradigm - get big random blocks by putting small random blocks together.
 	\begin{itemize}
 		\item let block size be 64
 		\item let the key specify 8 permutations(bijections) \ee{F = (f_1, f_2,...f_8)}, each with 8 bit block length
 		\item break the block \ee{X} into bytes \ee{(x_1, x_2,...x_8)}
 		\item \ee{F_k(x) = f_1(x_1)\|...\|f_{16}(x_{16})}
 		\item \ee{F} introduces \textit{confusion}
 		\item now apply a mixing permutation function to introduce \textit{diffusion}. If you don't, the for two different inputs that differ by one bit, the outputs will only differ in that one byte. 
 	\end{itemize}
 	\item S-box
 	\begin{itemize}
 		\item for SPN, instead of having some arbitrary function, instead have \ee{f} be a \textbf{PUBLIC} "substitution function" (permutation) \ee{S} called an S-box, and then let \ee{k} define the function \ee{f(x) = S(k \oplus x)}
 		\item evaluating proceeds in rounds. For each round
 		\begin{itemize}
 			\item Key mixing: \ee{x:=x \oplus k}, where \ee{k} is the current round sub-key
 			\item Substutiton: \ee{x:=S_1(x_1)\|...\|S_n(x_n)}, where \ee{x_i} is the \ee{i}th byte of \ee{x}
 			\item Permutation: apply a mixing permutation function to obtain output
 		\end{itemize}
 		after all rounds are complete, apply one more key mix, then output cipher block
 		\item SPNs are all invertible
 	\end{itemize}
 \end{itemize}

 \item[Feistel Network] \hfill
 \begin{itemize}
 	\item allows constructing ciphers(functions) from non-invertible components.
 	\item S-boxes need to be invertible, which is quite restrictive for functions.
 	\item for an input \ee{(L, R)}:
 	\subitem forward proceeds as:
 	\begin{itemize}
 		\item \ee{L_{i+1} = R_i}
 		\item \ee{R_{i+1} = L_i \oplus F(R_i, K_i)} 		
 	\end{itemize}
 	\subitem backwards:
 	\begin{itemize}
 		\item \ee{R_i = L_{i+1}}
 		\item \ee{L_i = R_{i+1} \oplus F(L_{i+1}, K_i)}
 	\end{itemize}
 	
 \end{itemize}

\newpage
 
\item[DES] \hfill
\begin{itemize}
 	\item DES is a 16 round feistel network
 	\item DES takes 64 bit block + 56 bit master key
 	\item the key schedule takes 16 48 bit subsets of the master key, \ee{k_1, k_2,...k_{16}}
 	\item DES round function is just a SPN. it takes 32 bit input (half block) \ee{R} + 48 bit key \ee{k_i}. \begin{enumerate}
 		\item \ee{R} is expanded to 48 bits by duplicating half the bits, denoted by \ee{R' = E(R)} where \ee{E} is our expansion function.
 		
 		\item \ee{R'} is then XORed with \ee{k_i} (also 48 bits long) and the result is divided into 8 6 bit blocks.
 		\begin{itemize}
 			\item DES S-box is just a 4x16 matrix, with each cell containing a 4 bit output, the first and last bits in the input map to row, the middle 4 map to column. 
 			\item S-box is a 4-1 function, with each input existing once in each row
 			\item One bit change to the input of S-box changes at least two bits in the output
 		\end{itemize}
 	
 		\item concatenate the output of the S-boxes
 		\item apply a mixing permutation to obtain the output
 		
 	\end{enumerate}
 	\item DES has avalanche effect
 	\begin{itemize}
 		\item suppose some input \ee{(L', R)}, and another input \ee{(L, R)}, in which \ee{L} differs from \ee{L'} by one bit. After one round, they still differ in one bit. After the output of the second S-box
	\end{itemize}

\end{itemize}

\item[Message authentication codes] \hfill
\begin{itemize}
	\item when sending message, compute a MAC tag, based on message + key
	\item \ee{{\mathit{Vrfy}}_k(m, \mathit{Mac}_k(M))=1}
	\item can be deterministic or random
	\begin{itemize}
		\item if deterministic, \ee{\mathit{Vrfy}} is just recomputing the tag
	\end{itemize}
	\item breaking MAC - Mac-forge
	\begin{itemize}
		\item if give an oracle for the \ee{\mathit{Mac}}, the adversary can output a message along with a tag, for one he hasn't seen before
	\end{itemize}
	\item existential unforgeability - unable to forge a valid tag on \textit{any} message
	\item protect against replay attacks with
	\begin{itemize}
		\item sequence numbers (=counters)
		\item timestamps
	\end{itemize}

	\item Strong Mac
	\begin{itemize}
		\item general Mac security means that given messages and tags, new tags for new messages cannot be computed, however, does not take into account the possibility of generating new tags for the messages already received
		\item net experiment called Mac-sforge
	\end{itemize}

	\item constructing Mac
	\begin 
\end{itemize}
  

\end{description}

\end{document}
 